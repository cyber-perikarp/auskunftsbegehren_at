\documentclass[%
  fontsize=12pt, % Schriftgröße
  version=last%  % Neueste Version von KOMA-Skript verwenden
]{scrlttr2}

\LoadLetterOption{template}

\begin{document}
\setkomavar{date}{\ort{}, den \today{}}
\begin{letter}
{
  \ziel{}
}

\opening{Sehr geehrte Damen und Herren,}
Ich wende mich an Sie, da Sie personenbezogene Daten über mich verwenden und speichern.\par
Gemäß Artikel 63 der EU-DSGVO i.V.m. § 44 DSG ersuche ich Sie um Auskunft über die über mich gespeicherten Daten, im Speziellen darüber:
\begin{itemize}
  \item den Inhalt der Datensätze,
  \item den Namen und die Kontaktdaten des Verantwortlichen,
  \item gegebenenfalls die Kontaktdaten des Datenschutzbeauftragten,
  \item die Zwecke, für die die personenbezogenen Daten verarbeitet werden,
  \item die Kontaktdaten der Aufsichtsbehörde,
  \item die Rechtsgrundlage der Verarbeitung,
  \item die Dauer, für die die personenbezogenen Daten gespeichert werden oder, falls dies nicht möglich ist, die Kriterien für die Festlegung dieser Dauer,
  \item gegebenenfalls die Kategorien von Empfängern der personenbezogenen Daten, auch der Empfänger in Drittländern oder in internationalen Organisationen,
  \item erforderlichenfalls weitere Informationen.
\end{itemize}

Ich ersuche Sie, auch Daten bekannt zu geben, die mit meinen Daten direkt oder indirekt verknüpft sind oder verknüpft werden können.\\
Sollten Daten einem Dienstleister überlassen worden sein, so ersuche ich um die Bekanntgabe des Namens und der Anschrift dieses Dienstleisters.\par
Zum Nachweis meiner Identität lege ich eine Kopie meines \ausweis{} bei.\\
Sollten noch Zweifel an meiner Identität bestehen, ersuche ich Sie mir die Auskunft per eingeschriebenem Brief zu eigenen Händen zuzustellen, da auch so die Identität überprüft werden kann.
Meiner Mitwirkungspflicht gemäß Artikel 64 der EU-DSGVO komme ich damit nach.\par

\ifx \zusatzinfo \undefined
\else
Als zusätzliche Informationen gebe ich bekannt:\\
\zusatzinfo{}
\fi

\textbf{Meine Anfrage bezieht sich auf alle Daten zu meiner Person.}\par

Gemäß § 42 (4) DSG ist die Auskunft innerhalb von vier Wochen nach Einlangen des Begehrens zu erteilen. Die Auskunft ist nach § 42 (6) DSG unentgeltlich zu erteilen.
\closing{Mit freundlichen Grüßen}

%\ps{PS:}
\encl{Kopie meines \ausweis{}}
%\cc{}

\end{letter}
\end{document}
